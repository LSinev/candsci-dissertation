\chapter*{Введение}							% Заголовок
\addcontentsline{toc}{chapter}{Введение}	% Добавляем его в оглавление

\newcommand{\actuality}{\textbf{\actualityTXT}} %Актуальность проблемы.
\newcommand{\progress}{}
\newcommand{\aim}{\textbf{\aimTXT}}
\newcommand{\tasks}{\tasksTXT}
\newcommand{\novelty}{\textbf{\noveltyTXT}}
\newcommand{\influence}{\textbf{\influenceTXT}}
\newcommand{\methods}{\textbf{\methodsTXT}}
\newcommand{\defpositions}{\textbf{\defpositionsTXT}}
\newcommand{\reliability}{\textbf{\reliabilityTXT}}
\newcommand{\probation}{\textbf{\probationTXT}}
\newcommand{\contribution}{\textbf{\contributionTXT}}
\newcommand{\publications}{\textbf{\publicationsTXT}}
\newcommand{\realisation}{\textbf{\realisationTXT}}

{\actuality}

Технология электростатического соединения пластин кремния и~стекла
(анодная посадка) стала одной из ключевых в производстве приборов
электронной техники. Одной из целей применения данной технологии является
изоляция чувствительного элемента от окружающей среды. Это касается
механических и электрических воздействий, сохранения герметичности
изолируемого объёма в таких приборах, как чувствительные элементы
датчиков давления, микроакселерометры, высокочувствительные
микромеханические гироскопы, высокочастотные резонаторы.

К преимуществам анодной посадки (по сравнению с~диффузионной сваркой,
и~соединением за~счёт расплавления стекла) относят: температуру
процесса ниже температуры деградации металлизации, возможность
сохранения герметичности соединения при использовании с~технологиями
формирования электрических межсоединений сквозь стекло или кремний,
отсутствие требования механического прижатия.

Вследствие разнородности кремния и стекла после проведения
их~соединения возникают остаточные напряжения, вызванные разницей
в~тепловом расширении этих материалов. Эти напряжения могут влиять
на такие ключевые характеристики, как:
уход нулевого сигнала
у чувствительных элементов датчиков давления, собственная частота
у микроакселерометров и микрогироскопов, геометрическое положение
мембран и~балок у~разнообразных микромеханических приборов.
Таким образом, необходимо, сохранив преимущества технологии сращивания
кремния и~стекла, учесть и~снизить влияние такого недостатка как
описанные остаточные напряжения.

Большой вклад в развитие исследований технологии электростатического
соединения кремния и стекла внесён советскими и российскими учёными
Н.\,Н.~Хоменко, Ю.\,М.~Евдокимовым, С.\,П.~Тимошенковым\ и~др.

\providecommand{\beforenedostati}%
{Несмотря на~большое количество исследований применения технологии
электростатического соединения кремния со~стеклом,
ряд вопросов остаётся невыясненным.
С повышением требований к~электронным приборам возросли требования
к минимизации остаточных напряжений.
}

\providecommand{\beforenedostatii}%
{Среди вопросов, остающихся нерешёнными,
стоит отметить следующие:}

\providecommand{\nedostati}% %с большой буквы и без точки в конце
{Недостаточно данных по~термомеханическим свойствам стёкол, подходящих
для электростатического соединения с кремнием, в форме, удобной
к применению в аналитических расчётах и в~системах компьютерного
моделирования}

\providecommand{\nedostatii}% %с большой буквы и без точки в конце
{Недостаточно хорошо исследован выбор температуры соединения
и~предпосылки наличия температуры проведения соединения
с минимальными остаточными напряжениями}

\providecommand{\nedostatiii}% %с большой буквы и без точки в конце
{Недостаточно подробно описано влияние соотношения толщин
соединяемых пластин на~возникающие после соединения напряжения}
%
\beforenedostati{}
\nedostati{}.
\nedostatii{}.
\nedostatiii{}.

Поэтому \MakeLowercase{\thesisTitleTXT}
является актуальной темой исследования.

\aim\ данной работы является
\MakeLowercase{\aimTextContent}.

Для~достижения поставленной цели необходимо было решить следующие {\tasks}:
\begin{enumerate}
\item \label{task1}\mmark{Исследовать зависимость} температурных коэффициентов линейного расширения
(\mmark{ТКЛР}) \mmark{от~температуры для стёкол}, совместимых с~анодной посадкой.
\item \label{task2}\mmark{Разработать модели оценки} остаточных напряжений, определить
их~возможности по учёту температурной зависимости ТКЛР, распределению
напряжений по~толщине материалов и~подбору температуры соединения
с~минимальными остаточными напряжениями.
\item \label{task3}\mmark{Разработать методику расчёта} остаточных напряжений
при использовании известных марок стёкол.
\item \label{task4}\mmark{Предложить способ коррекции} температуры
и~выбора толщины соединяемых деталей, обеспечивающих
минимальную величину остаточных напряжений.
\item \label{task5}Разработать методику и~\mmark{провести экспериментальные
исследования} процесса соединения кремния со~стеклом с~измерением
остаточных напряжений в~соединении.

\end{enumerate}

{\methods}
Теоретические исследования проводились на~основе теории напряжённого
состояния в~композиционных материалах, сопротивления материалов,
математического анализа.
В ходе исследований применялись расчёты напряжённо-деформированного
состояния сборок пластин кремния и~стекла с использованием
компьютерных программных пакетов.
Экспериментальные исследования осуществлялись
на~современных аналитических приборах, обработка результатов велась
с~помощью теории вероятностей и~математической статистики.

\novelty %Новизна
\begin{enumerate}
\item \label{novelty_one}%
Впервые \mmark{взаимосвязь температуры}
проведения соединения кремния со~стеклом и~\mmark{остаточных напряжений}
рассмотрена
\mmark{с~использованием температурной зависимости истинных значений}
температурных коэффициентов линейного расширения
материалов.
%Новая постановка известной задачи - приняты новые условия

\item \label{novelty_two}%Рябову нравится этот пункт:
Впервые доказано \mmark{существование соотношения толщин кремния
и~стекла}, минимизирующего остаточные напряжения на поверхности
кремния, за счёт взаимной компенсации деформаций вызванных
тепловым расширением кремния и~присоединяемого стекла.
\providecommand{\beforevyvods}%
{В диссертации поставлена и решена
научно-техническая задача
исследования методов снижения остаточных напряжений
при~электростатическом соединении кремния со~стеклом.}

%В заключении диссертации излагаются итоги выполненного исследования, рекомендации, перспективы дальнейшей разработки темы.

\newcommand{\vyvodi}%
{%
\item \label{vyvod_one}Расчётным путём выявлено, что для электростатического соединения
с минимальными остаточными напряжениями \mmark{подбор стекла}, согласованного
с кремнием по ТКЛР, следует осуществлять \mmark{по~критерию минимальности
накапливаемой разности между ТКЛР}, в~процессе охлаждения с температуры
соединения до диапазона рабочих температур прибора.
}

\newcommand{\vyvodii}%
{\item \label{vyvod_two}Для моделирования сборок
кремния со~стёклами марок
ЛК5, Borofloat~33, 7740, SD-2
в~диапазоне температур от~минус~100 до~500~{\textdegree}C
следует \mmark{использовать
полученные} автором диссертационной работы
\mmark{температурные зависимости ТКЛР} для этого диапазона температур.}

\newcommand{\vyvodiii}%
{\item \label{vyvod_three}Показано, что
\mmark{предварительный расчёт по моделям}
двух тонких слоёв и~многослойного композиционного материала
\mmark{с учётом динамики накопления}
остаточных напряжений
\mmark{повышает
эффективность} применения
моделирования методом \mmark{конечных элементов}.
}

\newcommand{\vyvodivmain}%
{\mmark{Для минимизации} остаточных \mmark{напряжений на поверхности} сплошного кремния необходимо \mmark{выбирать толщину стекла}:%
    \begin{itemize}
        \item алюмосиликатного (SD-2, SW-YY) в~2,5"--~2,8~раза больше
        толщины кремния.
        \item боросиликатного (ЛК5, 7740, Borofloat 33) в~3~раза больше
        толщины кремния.
    \end{itemize}
}
\newcommand{\vyvodiv}%
{\item \label{vyvod_four}\vyvodivmain
    Это объясняется разницей в согласованности жёсткости стекла и~кремния.
    Остаточные напряжения на поверхности кремния минимизируются за счёт компенсации деформаций, вызванных тепловым расширением кремния, деформациями, вызванными воздействием присоединённого стекла.%
}

\newcommand{\vyvodv}%
{\item \label{vyvod_five}\mmark{При охлаждении} сборки
со~скоростью
\mmark{не более
2~{\textdegree}C/мин}
после окончания подачи высокого напряжения
\mmark{происходит снижение остаточных напряжений} в стекле
(например в~ЛК5, соединённом
с~кремнием при температуре 440~{\textdegree}C,
до 63~\% от уровня
напряжений при неконтролируемом охлаждении).
Это объясняется
релаксацией аморфной структуры стекла.%
}

\newcommand{\vyvodvi}%
{\item \label{vyvod_six}\mmark{Рекомендуется проводить} электростатическое соединение
\mmark{в~режиме ограничения тока},
что снижает риски
локального перегрева границы кремний"--~стекло
и~последующего прожига стекла.%
}

\newcommand{\vyvodsall}{\vyvodi\vyvodii\vyvodiii\vyvodiv\vyvodv\vyvodvi}

\providecommand{\aftervyvods}%
{Результаты, полученные в диссертации, могут быть использованы при разработке
и~производстве приборов электронной техники с~использованием технологии
электростатического соединения, таких как микрорезонаторы, микрореле,
микроакселерометры, микрогироскопы, чувствительные элементы датчиков
давления.
Проведённые расчёты и сделанные выводы могут способствовать импортозамещению
зарубежных марок стекла в отечественных приборах электронной техники.

\textbf{Дальнейшая разработка темы}
может состоять в разработке аналитических моделей оценки распределения
остаточных напряжений в микрообработанном кремнии,
с~использованием температурной зависимости истинных значений ТКЛР
материалов. Кроме того, востребованными будут модели, учитывающие
изменение свойств стекла, связанное с переносом ионов в~результате
проведения процесса электростатического соединения. Для этого также
потребуется дополнительное исследование свойств стёкол как по составу,
так и~по~анализу связи подвижности ионов с~температурой и~подводимой
разностью потенциалов.}
%
\vyvodivmain

\item \label{novelty_three}Впервые предложено
\mmark{рассчитывать температуру} проведения соединения так, чтобы
\mmark{накопленная} за время остывания относительная \mmark{деформация была
минимальной},
что позволит минимизировать остаточные напряжения.
%Новые усовершенствованные критерии, показатели (и их обоснование)

\end{enumerate}

\influence\ %Практическая ценность работы
\begin{enumerate}
    \item Разработаны
    методики снижения
    остаточных напряжений, возникающих в технологии
    электростатического соединения кремния и~стекла, повышающие
    функциональные и~эксплуатационные характеристики приборов
    электронной техники.
    \item Установлена возможность использования инженерной
    методики расчёта для оценки возможностей снижения остаточных
    напряжений и повышения эффективности применения различных
    марок стекла при сращивании с кремнием.
\end{enumerate}

\defpositions %Основные положения и~результаты, выносимые на~защиту
\begin{enumerate}
\item \label{defposition_one}\mmark{Результаты исследования} зависимости \mmark{ТКЛР}
различных марок \mmark{стекла от температуры} в интервале от~минус~100
до~500~${}^\circ$C.
\item \label{defposition_two}\mmark{Методика минимизации}
остаточных \mmark{напряжений} за счёт \mmark{выбора марки стекла} для
электростатического соединения с~кремнием и~\mmark{режима} проведения
процесса.
\item \label{defposition_three}\mmark{Конструктивно-технологические
методы производства} чувствительных элементов приборов электронной техники
\mmark{при~наличии требований по~минимизации} механических напряжений,
возникающих в~результате применения технологии электростатического
соединения кремния со~стеклом.

\end{enumerate}

\reliability\ основывается на~проведённом комплексном анализе результатов теоретических данных и экспериментальных исследований.
Результаты экспериментов обработаны и подтверждены статистическими методами.
Сформулированные в диссертации научные положения, выводы и рекомендации обоснованы теоретическими решениями и экспериментальными данными и~не~противоречат известным положениям.

\probation\
Результаты работы докладывались и обсуждались на заседаниях кафедры
электронных технологий в машиностроении \mbox{МГТУ} им.~Н.Э.~Баумана,
на научно-технических конференциях «Молодёжь в~науке» ФГУП \mbox{<<РФЯЦ-ВНИИЭФ>>}
(г. Саров, октябрь 2008~г., октябрь 2014~г.),
на научно-технических конференциях молодых учёных ФГУП <<\mbox{ВНИИА} им.~Н.\,Л.~Духова>> (г. Москва,
март 2010~г.,
март 2013~г., март 2016~г.),
на~научно-технических конференциях молодых специалистов Росатома
<<Высокие технологии атомной отрасли. Молодёжь в~инновационном процессе>> (ФГУП
<<\mbox{ФНПЦ} \mbox{НИИИС} им. Ю.\,Е.~Седакова>>, г.~Нижний Новгород,
сентябрь 2015~г.
(доклад отмечен дипломом за~активную научную деятельность и~перспективные
исследования),
сентябрь 2016~г.),
на 6-ом международном МЭМС\nb-Форуме 2016
(Курский государственный университет, г.~Курск, июнь 2016~г.).

\contribution\ Диссертация является завершённой работой, в~которой обобщены
результаты исследований, полученные лично автором и~в~соавторстве.
Участие в работе каждого соавтора отражено в~совместных публикациях.
Совместно с научным руководителем был определён план работы,
разработаны основные теоретические положения главы~3.
Личный вклад автора включает:
проведение анализа современного состояния исследований
в~области использования технологии электростатического соединения кремния
со~стеклом;
разработку математических моделей, позволяющих прогнозировать
остаточные напряжения в зависимости от режима проведения
процесса сращивания кремния со~стеклом
и~их~термомеханических свойств;
проведение экспериментов по~измерению температурной
зависимости коэффициентов теплового расширения стёкол;
проведение экспериментов по~спектроскопии комбинационного рассеяния
света на~образцах соединённых пластин стекла и~кремния;
обработку экспериментальных данных
и формулировку рекомендаций по~использованию результатов работы.

%Сделана отдельная секция, чтобы не отображались в списке цитированных материалов
    \begin{refsection}[vak,papers,conf]%
        \printbibliography[heading=countauthornotvak, env=countauthornotvak, keyword=biblioauthornotvak, section=1]%
        \printbibliography[heading=countauthorvak, env=countauthorvak, keyword=biblioauthorvak, section=1]%
        \printbibliography[heading=countauthorconf, env=countauthorconf, keyword=biblioauthorconf, section=1]%
        \printbibliography[heading=countauthor, env=countauthor, keyword=biblioauthor, section=1]%
        \publications\ Основные результаты по теме диссертации изложены в~\arabic{citeauthor}~научных публикациях\nocite{Sinev_Ryabov_Tinyakov_nano2011, Sinev_osoben_primen_inzh_vest201408}
        общим объёмом 3,3~п.~л.,
        \arabic{citeauthorvak} из которых находятся в~изданиях из перечня
        рекомендованных ВАК РФ\nocite{Sinev_Ryabov_NMST_2011, Sinev_Ryabov_rasch_coef_napr_nmst2014, Sinev_technomag2014, Sinev_Petrov2016_cte_glass} (1,3~п.~л.),
        \arabic{citeauthorconf} "--- в тезисах докладов\nocite{sinev2008molodezh, sinev2010otsenka_vniia, sinev2013primenenie_vniia, sinev2014molodezh, sinev2015niiis, sinev2016issledovanie_vniia} (1,4~п.~л.).%
    \end{refsection}
    \begin{refsection}[vak,papers,conf]%
        \printbibliography[heading=countauthorvak, env=countauthorvak, keyword=biblioauthorvak, section=2]%
        \printbibliography[heading=countauthornotvak, env=countauthornotvak, keyword=biblioauthornotvak, section=2]%
        \printbibliography[heading=countauthorconf, env=countauthorconf, keyword=biblioauthorconf, section=2]%
        \printbibliography[heading=countauthor, env=countauthor, keyword=biblioauthor, section=2]%
        \nocite{Sinev_Ryabov_NMST_2011, Sinev_Ryabov_rasch_coef_napr_nmst2014, Sinev_technomag2014, Sinev_Petrov2016_cte_glass}%vak
        \nocite{Sinev_Ryabov_Tinyakov_nano2011, Sinev_osoben_primen_inzh_vest201408}%notvak
        \nocite{sinev2008molodezh, sinev2015niiis}%conf
    \end{refsection}
 % Характеристика работы по структуре во введении и в автореферате не отличается (ГОСТ Р 7.0.11, пункты 5.3.1 и 9.2.1), потому её загружаем из одного и того же внешнего файла, предварительно задав форму выделения некоторым параметрам

\textbf{Объём и структура работы.} Диссертация состоит из~введения,
четырёх глав, общих выводов и заключения.
Полный объём диссертации составляет \formbytotal{TotPages}{страниц}{у}{ы}{}
с~\formbytotal{totalcount@figure}{рисунк}{ом}{ами}{ами}
и~\formbytotal{totalcount@table}{таблиц}{ей}{ами}{ами}. Список литературы содержит
\formbytotal{citenum}{наименован}{ие}{ия}{ий}.

В первой главе представлены результаты анализа современного
состояния исследований в области применения технологий
герметичного соединения при изготовлении электронной техники.
Представлен обзор современных технологий, определены их~достоинства
и~недостатки.
Углублённо рассмотрены вопросы исследования и применения технологии
электростатического соединения кремния со~стеклом.

Вторая глава посвящена экспериментальному
исследованию зависимостей
температурных коэффициентов линейного расширения стёкол
от~температуры. Исследованы
термомеханические свойства и~состав основных марок отечественного
и зарубежного стекла, применяемого для электростатического соединения
с кремнием "--- ЛК5, Schott Borofloat~33, Corning~7740, Hoya~SD\nb-2.

В третьей главе описаны существующие модели расчётной оценки
остаточных напряжений в соединениях кремния со стеклом. Предложены две
новые модели на основании общей теории сопротивления материалов
и классической теории слоистых композитов с дополнительным учётом
температурной зависимости коэффициентов теплового расширения.
Приведены примеры применения этих моделей на исходных данных,
полученных в предшествующей главе. В завершении главы~3 даны
рекомендации по~последовательности проведения расчётной оценки
остаточных напряжений.

В четвёртой главе описаны эксперименты,
проведённые для подтверждения утверждений из~предыдущих глав. Описаны
исследования и практический опыт по снижению остаточных напряжений
посредством термообработки. Предложена методика минимизации остаточных
напряжений, представленная в виде шагов для разработчика прибора.
Также описаны сложности, на которые стоит обратить внимание при
разработке современной электронной техники с применением технологии
электростатического соединения.

В заключении приведены краткие выводы
по~результатам проведённых в~работе исследований.

\begingroup
Автор выражает благодарность И.\,Д.~Петрову и А.\,Ю.~Переяславцеву за~помощь в~организации и~проведении экспериментов по измерению свойств стёкол;
О.\,Л.~Лазаревой за помощь в~организации и~проведении спектроскопии методом комбинационного рассеяния света; М.\,Г.~Лукоперовой за помощь в~организации и~проведении поляриметрических измерений стекла;
Т.\,К.~Ерофеевой
за плодотворные дискуссии во время работы над диссертацией;
сотрудникам кафедры электронных технологий в~машиностроении \mbox{МГТУ} им.~Н.Э.~Баумана
за~ценные замечания по тексту диссертации и~автореферата.\russianpar
\endgroup
%