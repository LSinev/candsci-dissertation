\chapter*{Список сокращений}             % Заголовок
\addcontentsline{toc}{chapter}{Список сокращений}  % Добавляем его в оглавление
\noindent%
\addtocounter{table}{-1}% Нужно откатить на единицу счетчик номеров таблиц, так как следующая таблица сделана для удобства представления информации по ГОСТ
\begin{longtabu} to \textwidth {l X}

\textbf{КМОП} & комплементарная структура металл-оксид-полупроводник.\\

\textbf{КР} & комбинационное рассеяние.\\

\textbf{МЭМС} & микроэлектромеханическая система.\\

\textbf{ПИД-регулятор} & пропорционально-интегрально-дифференцирующий регулятор.\\
\textbf{РС} & Рамановская спектроскопия.\\
\textbf{РФЭС} & рентгеновская фотоэлектронная спектроскопия.\\

\textbf{ТКЛР} & температурный коэффициент линейного расширения.\\
\textbf{MSE} & среднеквадратичная ошибка (mean square error).\\

\textbf{RSS} & сумма квадратов регрессионных остатков (residual sum of~squares).\\

\textbf{SER} & стандартная ошибка регрессии (standard error of the regression).\\

\end{longtabu}
