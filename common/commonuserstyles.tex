%%%%%%%% Common User styles

\def\nb-{\nobreak\hskip0pt\hbox{-}\nobreak\hskip0pt}

%Корректное многоточие в LaTeX с равномерными промежутками между точками для тех, кто пишет в соответствии с правилами русского языка: %http://kostyrka.ru/blog/archives/883
%Если кому-то не хочется подключать пакет xspace, то предлагается самая робастная и компактная версия кода, в которой для обеспечения пробела после многоточия необходимо поставить пустую группу ({}):
\def\ellipsiskern{.1em}
\newcommand{\ldotst}{.\kern\ellipsiskern.\kern\ellipsiskern.} %... % \ldotst{}
\newcommand{\ldotse}{!\kern\ellipsiskern.\kern\ellipsiskern.} %!.. % \ldotse{}
\newcommand{\ldotsq}{?\kern\ellipsiskern\kern-.11em.\kern\ellipsiskern.} %?.. % \ldotsq{}

\renewcommand{\textdegree}{\ensuremath{{}^\circ}}
% Имеет смысл подумать о взаимодействии с пакетом repltext. В pdflatex в выдаче заменять на символ градуса, а в lua/xe(la)tex сразу символ градуса вписывать.

\sisetup{ %siunitx setup
    output-decimal-marker={,},
}

%%%% Русская традиция начертания математических знаков
\renewcommand{\le}{\ensuremath{\leqslant}}
\renewcommand{\leq}{\ensuremath{\leqslant}}
\renewcommand{\ge}{\ensuremath{\geqslant}}
\renewcommand{\geq}{\ensuremath{\geqslant}}
\renewcommand{\emptyset}{\varnothing}

%%%% Русская традиция начертания греческих букв (греческие буквы вертикальные, через пакет upgreek)
\renewcommand{\epsilon}{\ensuremath{\upvarepsilon}}   %  русская традиция записи
\renewcommand{\phi}{\ensuremath{\upvarphi}}
\renewcommand{\kappa}{\ensuremath{\varkappa}}
%%%
\renewcommand{\alpha}{\upalpha}
\renewcommand{\beta}{\upbeta}
\renewcommand{\gamma}{\upgamma}
\renewcommand{\delta}{\updelta}
\renewcommand{\varepsilon}{\upvarepsilon}
\renewcommand{\zeta}{\upzeta}
\renewcommand{\eta}{\upeta}
\renewcommand{\theta}{\uptheta}
\renewcommand{\vartheta}{\upvartheta}
\renewcommand{\iota}{\upiota}
\renewcommand{\kappa}{\upkappa}
\renewcommand{\lambda}{\uplambda}
\renewcommand{\mu}{\upmu}
\renewcommand{\nu}{\upnu}
\renewcommand{\xi}{\upxi}
\renewcommand{\pi}{\uppi}
\renewcommand{\varpi}{\upvarpi}
\renewcommand{\rho}{\uprho}
%\renewcommand{\varrho}{\upvarrho}
\renewcommand{\sigma}{\upsigma}
%\renewcommand{\varsigma}{\upvarsigma}
\renewcommand{\tau}{\uptau}
\renewcommand{\upsilon}{\upupsilon}
\renewcommand{\varphi}{\upvarphi}
\renewcommand{\chi}{\upchi}
\renewcommand{\psi}{\uppsi}
\renewcommand{\omega}{\upomega}

\usepackage{letltxmacro} %http://tex.stackexchange.com/a/47372/79756
\LetLtxMacro{\oldint}{\int}

\makeatletter
      \@ifpackagewith{wasysym}{integrals}
      {

      }{%
      \usepackage{scalerel} %http://tex.stackexchange.com/a/222280/79756
      \renewcommand{\int}{\mathop{\scalerel*{\rotatebox{12}{$\!\scriptstyle\oldint\!$}}{\oldint}}}

      }
\makeatother

\colorlet{siliconcolour}{black!30} % define color for silicon mass
\colorlet{glasscolour}{white} % define color for glass mass

\hyphenation{Coventor-ware}

\newcommand\blank[1][\textwidth]{\noindent\rule[-.2ex]{#1}{.4pt}}

\mathtoolsset{showonlyrefs=true} % Показывать номера только у тех формул, на которые есть \eqref{} в тексте.
%IMPORTANT Это фактически у меня вызвало баг, описанный в документации. Во-первых всё, что без eqref лишается нумерации, а такие места у меня есть в третьей главе, что при ссылках, тоже выдает неверно потом. И для слишком широких формул. Unfortunately the use of the showonlyref introduce a bug within amsmath’s typesetting of formula versus equation number.

\newcommand\mmark{} %Специальное выделение для блоков текста