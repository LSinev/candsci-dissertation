%%% Основные сведения %%%
\newcommand{\AuthorURL}{http://orcid.org/0000-0003-2097-9036}
\newcommand{\thesisAuthor}             % Диссертация, ФИО автора
{%
    \texorpdfstring{%
        \hypersetup{urlcolor=black}% гарантия того, что ссылка не раскрашена будет
        \href{\AuthorURL}{Синев Леонид Станиславович}%
        \hypersetup{urlcolor={urlcolor}}%
    }{%
        Леонид Станиславович Синев%
    }%
}

%TITLE
\providecommand{\thesisTitleTXT}{Расчёт и~выбор режимов электростатического соединения кремния со~стеклом по~критерию минимума остаточных напряжений}
\newcommand{\thesisTitle}              % Диссертация, название
{%
    \thesisTitleTXT%
}
\newcommand{\thesisSpecialtyNumber}    % Диссертация, специальность, номер
{05.27.06}
\newcommand{\thesisSpecialtyTitle}     % Диссертация, специальность, название
{Технология и оборудование для~производства полупроводников, материалов и~приборов электронной техники}
\newcommand{\thesisDegree}             % Диссертация, ученая степень
{кандидата технических наук}
\newcommand{\thesisDegreeShort}        % Диссертация, ученая степень, краткая запись
{канд. техн. наук}
\newcommand{\thesisCity}               % Диссертация, город защиты
{Москва}
\newcommand{\thesisYear}               % Диссертация, год защиты
{2016}
\newcommand{\thesisOrganization}       % Диссертация, организация
{Московский государственный технический университет им.~Н.Э.~Баумана}

\newcommand{\thesisInOrganization}     % Диссертация, организация в предложном падеже: Работа выполнена в ...
{Московском государственном техническом
университете им.~Н.Э.~Баумана}

\newcommand{\supervisorFio}            % Научный руководитель, ФИО
{%
    \texorpdfstring{%
        \hypersetup{urlcolor=black}% гарантия того, что ссылка не раскрашена будет
        \href{http://elibrary.ru/author_items.asp?authorid=570330}{Рябов Владимир Тимофеевич}%
        \hypersetup{urlcolor={urlcolor}}%
    }{%
        Рябов Владимир Тимофеевич%
    }%
}
\newcommand{\supervisorRegalia}        % Научный руководитель, регалии
{кандидат технических наук, доцент}
\newcommand{\supervisorFioShort}       % Научный руководитель, ФИО
{В.\,Т.\,Рябов}

\newcommand{\opponentOneFio}           % Оппонент 1, ФИО
{%
    \hypersetup{urlcolor=black}% гарантия того, что ссылка не раскрашена будет
    \href{http://elibrary.ru/author_items.asp?authorid=22067}{Амиров Ильдар Искандерович}%
    \hypersetup{urlcolor={urlcolor}}%
}
\newcommand{\opponentOneRegalia}       % Оппонент 1, регалии
{доктор физико-математических наук}
\newcommand{\opponentOneJobPlace}      % Оппонент 1, место работы
{Ярославского Филиала Федерального государственного бюджетного учреждения науки Физико-технологического института Российской академии наук (ЯФ ФТИАН РАН)}
\newcommand{\opponentOneJobPost}       % Оппонент 1, должность
{заместитель директора по научной работе}

\newcommand{\opponentTwoFio}           % Оппонент 2, ФИО
{%
    \hypersetup{urlcolor=black}% гарантия того, что ссылка не раскрашена будет
    \href{http://elibrary.ru/author_items.asp?authorid=852587}{Жукова Светлана Александровна}%
    \hypersetup{urlcolor={urlcolor}}%
}
\newcommand{\opponentTwoRegalia}       % Оппонент 2, регалии
{кандидат технических наук}
\newcommand{\opponentTwoJobPlace}      % Оппонент 2, место работы
{ФГУП <<Центральный научно-исследовательский институт химии и механики>> (ФГУП <<ЦНИИХМ>>)}
\providecommand{\opponentTwoJobPost}       % Оппонент 2, должность
{заместитель начальника центра~--- начальник научно-технологического комплекса нано- и~микротехнологий НИЦ нанотехнологий}

\newcommand{\leadingOrganizationTitle} % Ведущая организация, дополнительные строки
{Федеральное государственное автономное образовательное учреждение высшего образования «Национальный исследовательский университет «Московский институт электронной техники» (МИЭТ)}

\newcommand{\defenseDate}              % Защита, дата
{<<\blank[\widthof{881}]>>\blank[\widthof{сенсентября}] 201\blank[\widthof{81}]~г.~в~\blank[\widthof{881}]~ч. \blank[\widthof{881}]~мин.}
\newcommand{\defenseCouncilNumber}     % Защита, номер диссертационного совета
{Д\,212.141.18}
\newcommand{\defenseCouncilTitle}      % Защита, учреждение диссертационного совета
{Московском государственном техническом университете им.~Н.Э.~Баумана}
\newcommand{\defenseCouncilAddress}    % Защита, адрес учреждение диссертационного совета
{105005, г.~Москва, ул.~2-я Бауманская д.~5, стр.~1}
\newcommand{\defenseCouncilPhone}      % Телефон для справок
{+7~(499)~267-09-63}

\newcommand{\defenseSecretaryFio}      % Секретарь диссертационного совета, ФИО
{Цветков Ю.\,Б.}
\newcommand{\defenseSecretaryRegalia}  % Секретарь диссертационного совета, регалии
{доктор технических наук, профессор}   % Для сокращений есть ГОСТы, например: ГОСТ Р 7.0.12-2011 + http://base.garant.ru/179724/#block_30000

\newcommand{\synopsisLibrary}          % Автореферат, название библиотеки
{МГТУ им.~Н.Э.~Баумана и~на~сайте
\hypersetup{urlcolor=black}% гарантия того, что ссылка не раскрашена будет
\href{http://www.bmstu.ru/mstu/works/science/degree-candidates/dissertants/?q=dissertation&id=220}{www.bmstu.ru}%
\hypersetup{urlcolor={urlcolor}}%
}
\newcommand{\synopsisDate}             % Автореферат, дата рассылки
{<<\blank[\widthof{888}]>>\blank[\widthof{сенсентября}] 201\blank[\widthof{81}]~года}

% To avoid conflict with beamer class use \providecommand
\providecommand{\keywords}%            % Ключевые слова для метаданных PDF диссертации и автореферата
{анодная посадка, электростатическое сращивание, электростатическое соединение,
электроадгезионное соединение, термоэлектростимулированное соединение, кремний,
боросиликатное стекло, алюмосиликатное стекло, температурный коэффициент
линейного расширения, ТКЛР, остаточные напряжения, коэффициентные напряжения,
anodic bonding, residual stress, thermal stress, wafer bonding, CTE,
borosilicate glass}

\providecommand{\aimTextContent}%       % цель в именительном падеже с большой буквы
{Научно обоснованный выбор эффективных режимов анодной посадки кремния на~стекло, обеспечивающих минимальный уровень остаточных напряжений}
\providecommand{\aimTextContentRod}%    % цель в родительном (?) падеже с маленькой буквы
{научно обоснованного выбора эффективных режимов анодной посадки кремния на~стекло, обеспечивающих минимальный уровень остаточных напряжений}
