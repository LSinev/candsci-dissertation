\providecommand{\beforevyvods}%
{В диссертации поставлена и решена
научно-техническая задача
исследования методов снижения остаточных напряжений
при~электростатическом соединении кремния со~стеклом.}

%В заключении диссертации излагаются итоги выполненного исследования, рекомендации, перспективы дальнейшей разработки темы.

\newcommand{\vyvodi}%
{%
\item \label{vyvod_one}Расчётным путём выявлено, что для электростатического соединения
с минимальными остаточными напряжениями \mmark{подбор стекла}, согласованного
с кремнием по ТКЛР, следует осуществлять \mmark{по~критерию минимальности
накапливаемой разности между ТКЛР}, в~процессе охлаждения с температуры
соединения до диапазона рабочих температур прибора.
}

\newcommand{\vyvodii}%
{\item \label{vyvod_two}Для моделирования сборок
кремния со~стёклами марок
ЛК5, Borofloat~33, 7740, SD-2
в~диапазоне температур от~минус~100 до~500~{\textdegree}C
следует \mmark{использовать
полученные} автором диссертационной работы
\mmark{температурные зависимости ТКЛР} для этого диапазона температур.}

\newcommand{\vyvodiii}%
{\item \label{vyvod_three}Показано, что
\mmark{предварительный расчёт по моделям}
двух тонких слоёв и~многослойного композиционного материала
\mmark{с учётом динамики накопления}
остаточных напряжений
\mmark{повышает
эффективность} применения
моделирования методом \mmark{конечных элементов}.
}

\newcommand{\vyvodivmain}%
{\mmark{Для минимизации} остаточных \mmark{напряжений на поверхности} сплошного кремния необходимо \mmark{выбирать толщину стекла}:%
    \begin{itemize}
        \item алюмосиликатного (SD-2, SW-YY) в~2,5"--~2,8~раза больше
        толщины кремния.
        \item боросиликатного (ЛК5, 7740, Borofloat 33) в~3~раза больше
        толщины кремния.
    \end{itemize}
}
\newcommand{\vyvodiv}%
{\item \label{vyvod_four}\vyvodivmain
    Это объясняется разницей в согласованности жёсткости стекла и~кремния.
    Остаточные напряжения на поверхности кремния минимизируются за счёт компенсации деформаций, вызванных тепловым расширением кремния, деформациями, вызванными воздействием присоединённого стекла.%
}

\newcommand{\vyvodv}%
{\item \label{vyvod_five}\mmark{При охлаждении} сборки
со~скоростью
\mmark{не более
2~{\textdegree}C/мин}
после окончания подачи высокого напряжения
\mmark{происходит снижение остаточных напряжений} в стекле
(например в~ЛК5, соединённом
с~кремнием при температуре 440~{\textdegree}C,
до 63~\% от уровня
напряжений при неконтролируемом охлаждении).
Это объясняется
релаксацией аморфной структуры стекла.%
}

\newcommand{\vyvodvi}%
{\item \label{vyvod_six}\mmark{Рекомендуется проводить} электростатическое соединение
\mmark{в~режиме ограничения тока},
что снижает риски
локального перегрева границы кремний"--~стекло
и~последующего прожига стекла.%
}

\newcommand{\vyvodsall}{\vyvodi\vyvodii\vyvodiii\vyvodiv\vyvodv\vyvodvi}

\providecommand{\aftervyvods}%
{Результаты, полученные в диссертации, могут быть использованы при разработке
и~производстве приборов электронной техники с~использованием технологии
электростатического соединения, таких как микрорезонаторы, микрореле,
микроакселерометры, микрогироскопы, чувствительные элементы датчиков
давления.
Проведённые расчёты и сделанные выводы могут способствовать импортозамещению
зарубежных марок стекла в отечественных приборах электронной техники.

\textbf{Дальнейшая разработка темы}
может состоять в разработке аналитических моделей оценки распределения
остаточных напряжений в микрообработанном кремнии,
с~использованием температурной зависимости истинных значений ТКЛР
материалов. Кроме того, востребованными будут модели, учитывающие
изменение свойств стекла, связанное с переносом ионов в~результате
проведения процесса электростатического соединения. Для этого также
потребуется дополнительное исследование свойств стёкол как по составу,
так и~по~анализу связи подвижности ионов с~температурой и~подводимой
разностью потенциалов.}
