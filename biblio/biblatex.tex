%%% Реализация библиографии пакетами biblatex и biblatex-gost с использованием движка biber %%%

\usepackage{csquotes} % biblatex рекомендует его подключать. Пакет для оформления сложных блоков цитирования.

%%% Загрузка пакета с основными настройками %%%
\ifnumequal{\value{draft}}{0}{% Чистовик
\usepackage[%
backend=biber,% движок
bibencoding=utf8,% кодировка bib файла
sorting=none,% настройка сортировки списка литературы
style=gost-numeric,% стиль цитирования и библиографии (по ГОСТ)
language=autobib,% получение языка из babel/polyglossia, default: autobib % если ставить autocite или auto, то цитаты в тексте с указанием страницы, получат указание страницы на языке оригинала
autolang=other,% многоязычная библиография
clearlang=true,% внутренний сброс поля language, если он совпадает с языком из babel/polyglossia
defernumbers=true,% нумерация проставляется после двух компиляций, зато позволяет выцеплять библиографию по ключевым словам и нумеровать не из большего списка
sortcites=true,% сортировать номера затекстовых ссылок при цитировании (если в квадратных скобках несколько ссылок, то отображаться будут отсортированно, а не абы как)
doi=false,% Показывать или нет ссылки на DOI
isbn=false,% Показывать или нет ISBN, ISSN, ISRN
]{biblatex}[2016/09/17]%
}{%Черновик
\usepackage[%
backend=biber,% движок
bibencoding=utf8,% кодировка bib файла
sorting=none,% настройка сортировки списка литературы
style=gost-numeric,% стиль цитирования и библиографии (по ГОСТ)
]{biblatex}[2016/09/17]%
}

%%% Подключение файлов bib %%%
\addbibresource[label=other]{biblio/othercites.bib}
\addbibresource[label=vak]{biblio/authorpapersVAK.bib}
\addbibresource[label=papers]{biblio/authorpapers.bib}
\addbibresource[label=conf]{biblio/authorconferences.bib}


%http://tex.stackexchange.com/a/141831/79756
%There is a way to automatically map the language field to the langid field. The following lines in the preamble should be enough to do that.
%This command will copy the language field into the langid field and will then delete the contents of the language field. The language field will only be deleted if it was successfully copied into the langid field.
\makeatletter
\ltx@iffilelater{biblatex-gost.def}{2017/01/28}% biblatex-gost 1.11a
  {%
    \newcommand{\conditionalsm}{%
        \map[overwrite]{ % удаление "наук", чтобы biblatex-gost 1.11a не пытался его дублировать
            \pertype{thesis}
            \step[fieldsource=major,
            match=\regexp{(\\\Wнаук)},
            replace={}, %русские буквы просто так не вставить сюда
            final]
        }%
    }
  }{%
    \newcommand{\conditionalsm}{}
  }
\ltx@iffilelater{biblatex-gost.def}{2017/02/05}% biblatex-gost 1.12
  {%
    \renewcommand{\conditionalsm}{}
  }{%
  }
\makeatother
\DeclareSourcemap{ %модификация bib файла перед тем, как им займётся biblatex 
    \maps{
        \conditionalsm
        \map{% перекидываем значения полей language в поля langid, которыми пользуется biblatex
            \step[fieldsource=language, fieldset=langid, origfieldval, final]
            \step[fieldset=language, null]
        }
        \map[overwrite, refsection=0]{% стираем значения всех полей addendum
            \perdatasource{biblio/authorpapersVAK.bib}
            \perdatasource{biblio/authorpapers.bib}
            \perdatasource{biblio/authorconferences.bib}
            \step[fieldsource=addendum, final]
            \step[fieldset=addendum, null] %чтобы избавиться от информации об объёме авторских статей, в отличие от автореферата
        }
        \map[overwrite]{
            \perdatasource{biblio/authorpapersVAK.bib}
            \perdatasource{biblio/authorpapers.bib}
            \step[fieldsource=urldate, final]
            \step[fieldset=urldate, null]
        }
        \map[overwrite]{ % добавляем ключевые слова, чтобы различать источники
            \perdatasource{biblio/othercites.bib}
            \step[fieldset=keywords, fieldvalue={biblioother,bibliofull}]
        }
        \map[overwrite]{ % добавляем ключевые слова, чтобы различать источники
            \perdatasource{biblio/authorpapersVAK.bib}
            \step[fieldset=keywords, fieldvalue={biblioauthorvak,biblioauthor,bibliofull}]
        }
        \map[overwrite]{ % добавляем ключевые слова, чтобы различать источники
            \perdatasource{biblio/authorpapers.bib}
            \step[fieldset=keywords, fieldvalue={biblioauthornotvak,biblioauthor,bibliofull}]
        }
        \map[overwrite]{ % добавляем ключевые слова, чтобы различать источники
            \perdatasource{biblio/authorconferences.bib}
            \step[fieldset=keywords, fieldvalue={biblioauthorconf,biblioauthor,bibliofull}]
        }
        % Так отключаем [Электронный ресурс]
        \map[overwrite]{% стираем значения всех полей media=eresource
            \step[fieldsource=media,
            match={eresource},
            final]
            \step[fieldset=media, null]
        }
    }
}

%%% Правка записей типа thesis, чтобы дважды не писался автор
\ifnumequal{\value{draft}}{0}{% Чистовик
\DeclareBibliographyDriver{thesis}{%
  \usebibmacro{bibindex}%
  \usebibmacro{begentry}%
  \usebibmacro{heading}%
  \newunit
  \usebibmacro{author}%
  \setunit*{\labelnamepunct}%
  \usebibmacro{thesistitle}%
  \setunit{\respdelim}%
  %\printnames[last-first:full]{author}%Вот эту строчку нужно убрать, чтобы автор диссертации не дублировался
  \newunit\newblock
  \printlist[semicolondelim]{specdata}%
  \newunit
  \usebibmacro{institution+location+date}%
  \newunit\newblock
  \usebibmacro{chapter+pages}%
  \newunit
  \printfield{pagetotal}%
  \newunit\newblock
  \usebibmacro{doi+eprint+url+note}%
  \newunit\newblock
  \usebibmacro{addendum+pubstate}%
  \setunit{\bibpagerefpunct}\newblock
  \usebibmacro{pageref}%
  \newunit\newblock
  \usebibmacro{related:init}%
  \usebibmacro{related}%
  \usebibmacro{finentry}}
}{}

\newbibmacro{string+doi}[1]{% новая макрокоманда на простановку ссылки на doi
    \iffieldundef{doi}{#1}{\href{http://dx.doi.org/\thefield{doi}}{#1}}}

\ifnumequal{\value{draft}}{0}{% Чистовик
\renewcommand*{\mkgostheading}[1]{\usebibmacro{string+doi}{#1}} % ссылка на doi с авторов. стоящих впереди записи
}{}
\DeclareFieldFormat{title}{\usebibmacro{string+doi}{#1}} % ссылка на doi с названия работы
\DeclareFieldFormat{journaltitle}{\usebibmacro{string+doi}{#1}} % ссылка на doi с названия журнала
%%% Убрать тире из разделителей элементов в библиографии:
\renewcommand*{\newblockpunct}{%
    \addperiod\space\bibsentence}%block punct.,\bibsentence is for vol,etc.

%%% Исправление длины тире в диапазонах %%%
\makeatletter
\@ifpackagelater{biblatex}{2016/05/11}% biblatex 3.5+
  {%
    \DefineBibliographyExtras{russian}{%
      \protected\def\bibrangedash{%
        \textendash\penalty\value{abbrvpenalty}}% almost unbreakable dash
      \protected\def\bibdaterangesep{\bibrangedash}%тире для дат
    }%
  }
  {%
    \DefineBibliographyExtras{russian}{%
      \protected\def\bibrangedash{%
        \textendash\penalty\value{abbrvpenalty}}% almost unbreakable dash
      \protected\def\bibdatedash{\bibrangedash}%тире для дат
    }%
  }
\makeatother

\makeatletter
\@ifpackagelater{biblatex}{2016/05/11}% biblatex 3.5+
  {%
    \DefineBibliographyExtras{english}{%
      \protected\def\bibrangedash{%
        \textendash\penalty\value{abbrvpenalty}}% almost unbreakable dash
      \protected\def\accdate{\mainlang дата\addabbrvspace обр\adddot}%
      \restorecommand\mkdaterangecomp% several lines to revert internal US reverse date format modification
      \restorecommand\mkdaterangecompextra
      \restorecommand\mkdaterangeterse
      \restorecommand\mkdaterangeterseextra
      % d-m-y format for long dates
      \protected\def\mkbibdatelong#1#2#3{%
        \iffieldundef{#3}
          {}
          {\stripzeros{\thefield{#3}}%
           \iffieldundef{#2}{}{\nobreakspace}}%
        \iffieldundef{#2}
          {}
          {\mkbibmonth{\thefield{#2}}%
           \iffieldundef{#1}{}{\space}}%
        \iffieldbibstring{#1}{\bibstring{\thefield{#1}}}{\stripzeros{\thefield{#1}}}}%
      % d-m-y format for short dates
      \protected\def\mkbibdateshort#1#2#3{%
        \iffieldundef{#3}
          {}
          {\mkdayzeros{\thefield{#3}}%
           \iffieldundef{#2}{}{\adddot}}%
        \iffieldundef{#2}
          {}
          {\mkmonthzeros{\thefield{#2}}%
           \iffieldundef{#1}{}{\adddot}}%
        \iffieldbibstring{#1}{\bibstring{\thefield{#1}}}{\mkyearzeros{\thefield{#1}}}}%
    }%
  }
  {%
    \DefineBibliographyExtras{english}{%
      \protected\def\bibrangedash{%
        \textendash\penalty\value{abbrvpenalty}}% almost unbreakable dash
      \protected\def\accdate{\mainlang дата\addabbrvspace обр\adddot}%
      \restorecommand\mkbibrangecomp% several lines to revert internal US reverse date format modification
      \restorecommand\mkbibrangecompextra
      \restorecommand\mkbibrangeterse
      \restorecommand\mkbibrangeterseextra
      % d-m-y format for long dates
      \protected\def\mkbibdatelong#1#2#3{%
        \iffieldundef{#3}
          {}
          {\stripzeros{\thefield{#3}}%
           \iffieldundef{#2}{}{\nobreakspace}}%
        \iffieldundef{#2}
          {}
          {\mkbibmonth{\thefield{#2}}%
           \iffieldundef{#1}{}{\space}}%
        \iffieldbibstring{#1}{\bibstring{\thefield{#1}}}{\stripzeros{\thefield{#1}}}}%
      % d-m-y format for short dates
      \protected\def\mkbibdateshort#1#2#3{%
        \iffieldundef{#3}
          {}
          {\mkdatezeros{\thefield{#3}}%
           \iffieldundef{#2}{}{\adddot}}%
        \iffieldundef{#2}
          {}
          {\mkdatezeros{\thefield{#2}}%
           \iffieldundef{#1}{}{\adddot}}%
        \iffieldbibstring{#1}{\bibstring{\thefield{#1}}}{\mkdatezeros{\thefield{#1}}}}%
    }%
  }
\makeatother

%Set higher penalty for breaking in number, dates and pages ranges
\setcounter{abbrvpenalty}{10000} % default is \hyphenpenalty which is 12

%Set higher penalty for breaking in names
\setcounter{highnamepenalty}{10000} % If you prefer the traditional BibTeX behavior (no linebreaks at highnamepenalty breakpoints), set it to ‘infinite’ (10 000 or higher).
\setcounter{lownamepenalty}{10000}

%%% Set low penalties for breaks at uppercase letters and lowercase letters
\setcounter{biburllcpenalty}{500} %управляет разрывами ссылок после маленьких букв RTFM biburllcpenalty
\setcounter{biburlucpenalty}{3000} %управляет разрывами ссылок после больших букв, RTFM biburlucpenalty

\DeclareNumChars*{ ()}

\NewBibliographyString{yearsign}
\DefineBibliographyStrings{english}{%
yearsign = {},
urlseen = {\accdate},
pages = {P\adddot},%NEEDFIX здесь должна быть маленькая буква p, но залитый в ВАК диссер изменять нельзя, так что это не подлежит исправлению в исходниках моей диссертации
}
\DefineBibliographyStrings{russian}{%
yearsign = {\addnbspace г\adddot},
page = {C\adddot},
}

\renewbibmacro*{booktitle}{%
  \ifboolexpr{
    test {\iffieldundef{booktitle}}
    and
    test {\iffieldundef{booksubtitle}}
  }
    {}
    {\printtext[booktitle]{%
       \printfield[titlecase]{booktitle}%
       \setunit{\subtitlepunct}%
       \printfield[titlecase]{booksubtitle}}%
     \newunit}%
  \ifboolexpr{
    test {\iffieldundef{eventtitle}}
    and
    not test {\iffieldundef{eventyear}}
  }
    {\setunit{\addspace}%
     \printtext[parens]{%
       \printfield{venue}%
       \setunit*{\addcomma\space}%
       \printeventdate%
       \bibstring{yearsign}%
     }%
    }
    {}
  \setunit{\addcolondelim}%
  \printfield{booktitleaddon}}

\renewbibmacro*{event+venue+date}{%
    \ifboolexpr{
      test {\iffieldundef{eventtitle}}
      or
      (
        test {\iffieldundef{venue}}
        and
        test {\iffieldundef{eventyear}}
      )
    }
      {}
      {\setunit{\addspace}%
       \printtext[parens]{%
         \printfield{eventtitle}%
         \setunit{\addcolondelim}%
         \printfield{eventtitleaddon}%
         \setunit{\addcomma\space}%
         \printfield{venue}%
         \setunit*{\addcomma\space}%
         \printeventdate%
         \iffieldundef{eventyear}{}{\bibstring{yearsign}}%
       }}%
  \newunit}

\makeatletter
\ltx@iffilelater{biblatex-gost.def}{2017/02/05}% biblatex-gost 1.12
  {%
    \renewcommand{\doublevolsdelim}{\textendash\penalty\value{abbrvpenalty}}
  }{%
  }
\makeatother

%% Счётчик использованных ссылок на литературу, обрабатывающий с учётом неоднократных ссылок
%http://tex.stackexchange.com/a/66851/79756
\newtotcounter{citenum}
\makeatletter
\defbibenvironment{counter} %Env of bibliography
  {\setcounter{citenum}{0}%
  \renewcommand{\blx@driver}[1]{}%
  } %what is doing at the beginining of bibliography. In your case it's : a. Reset counter b. Say to print nothing when a entry is tested.
  {} %Здесь то, что будет выводиться командой \printbibliography. \thecitenum сюда писать не надо
  {\stepcounter{citenum}} %What is printing / executed at each entry.
\makeatother
\defbibheading{counter}{}

\newtotcounter{citeauthorvak}
\makeatletter
\defbibenvironment{countauthorvak} %Env of bibliography
{\setcounter{citeauthorvak}{0}%
    \renewcommand{\blx@driver}[1]{}%
} %what is doing at the beginining of bibliography. In your case it's : a. Reset counter b. Say to print nothing when a entry is tested.
{} %Здесь то, что будет выводиться командой \printbibliography. Обойдёмся без \theciteauthorvak в нашей реализации
{\stepcounter{citeauthorvak}} %What is printing / executed at each entry.
\makeatother
\defbibheading{countauthorvak}{}

\newtotcounter{citeauthornotvak}
\makeatletter
\defbibenvironment{countauthornotvak} %Env of bibliography
{\setcounter{citeauthornotvak}{0}%
    \renewcommand{\blx@driver}[1]{}%
} %what is doing at the beginining of bibliography. In your case it's : a. Reset counter b. Say to print nothing when a entry is tested.
{} %Здесь то, что будет выводиться командой \printbibliography. Обойдёмся без \theciteauthornotvak в нашей реализации
{\stepcounter{citeauthornotvak}} %What is printing / executed at each entry.
\makeatother
\defbibheading{countauthornotvak}{}

\newtotcounter{citeauthorconf}
\makeatletter
\defbibenvironment{countauthorconf} %Env of bibliography
{\setcounter{citeauthorconf}{0}%
    \renewcommand{\blx@driver}[1]{}%
} %what is doing at the beginining of bibliography. In your case it's : a. Reset counter b. Say to print nothing when a entry is tested.
{} %Здесь то, что будет выводиться командой \printbibliography. Обойдёмся без \theciteauthorconf в нашей реализации
{\stepcounter{citeauthorconf}} %What is printing / executed at each entry.
\makeatother
\defbibheading{countauthorconf}{}

\newtotcounter{citeauthor}
\makeatletter
\defbibenvironment{countauthor} %Env of bibliography
{\setcounter{citeauthor}{0}%
    \renewcommand{\blx@driver}[1]{}%
} %what is doing at the beginining of bibliography. In your case it's : a. Reset counter b. Say to print nothing when a entry is tested.
{} %Здесь то, что будет выводиться командой \printbibliography. Обойдёмся без \theciteauthor в нашей реализации
{\stepcounter{citeauthor}} %What is printing / executed at each entry.
\makeatother
\defbibheading{countauthor}{}

\defbibheading{authorpublications}[\authorbibtitle]{\section*{#1}}
\defbibheading{otherpublications}{\section*{#1}}


%%% Создание команд для вывода списка литературы %%%
\newcommand*{\insertbibliofull}{%
\printbibliography[keyword=bibliofull,section=0]
\printbibliography[heading=counter,env=counter,keyword=bibliofull,section=0]%
}

\newcommand*{\insertbiblioauthor}{
\printbibliography[heading=authorpublications,keyword=biblioauthor,section=1,title=\authorbibtitle]
\printbibliography[heading=counter,env=counter,keyword=biblioauthor,section=1]
}
\newcommand*{\insertbiblioauthorimportant}{
\printbibliography[heading=authorpublications,keyword=biblioauthor,section=2,title={Наиболее значимые \MakeLowercase{\protect\authorbibtitle{}}}]
\printbibliography[heading=counter,env=counter,keyword=biblioauthor,section=2]
}

\newcommand*{\insertbiblioother}{
\printbibliography[heading=otherpublications,keyword=biblioother]
\printbibliography[heading=counter,env=counter,keyword=biblioother]
}
